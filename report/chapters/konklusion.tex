\chapter{Konklusion}

Bookie er blevet både meget brugervenligt og brugbart. Det er praktiskt, kører hurtigt og er nemt at bruge. Der er muligt for ekspedienten at reservere pladser med kundens telefonnummer som id, redigere og slette, og have et overblik over alle reservationerne. 

Selv om programmet virker, kan der godt tilføjes nogle ændringer, for at få det til at virke optimalt. Vi nævner nogle forbedringer, hvilke bl.a. ville gøre Bookie mere brugbart i en reel biograf.

\begin{itemize}
  \item Muligheder for køb og salg af billetter.
  
  \item Reservation over nettet.
  
  \item Flere billetluger.
  
  \item Det skal ikke være muligt at indtaste ugyldige telefonnumre, som f.eks. \textit{
  11222222} eller \textit{11400000}.
  %I dette tilfælde ville det være alarmcentralen og politiet, som det ville   gå ud over. Chancen for at nogen ville brokke sig, er rimelig stor. Hvilket ikke havde været så godt.
  \item At se hvor mange ledige ud af det samlede antal af pladser der er tilbage.
  \item Mulighed for at vælge sæder mere effektivt.
  % F.eks. at indtaste det antal af sæder man vil reservere, hvorefter det antal af sæder ville blive markeret, og man derefter ville kunne vælge hvilken placering de skulle have.
  
  \item Tastaturunderstøttelse.
  %I tabellerne er det muligt at navigere med piletasterne, da dette kommer automatisk med JavaFX, men f.eks. at vælge sæder med pilene er indtil videre ikke muligt.
  

\end{itemize}
Der er tilføjet nogle ændringer, som ikke var krav til programmet. Men køb og salg af billetter, samt reservationer over nettet, er en nødvendighed for at få et optimalt reservationssystem at virke som det skal. 

Selv om disse ændringer ville gøre Bookie til et bedre system, og ekspedienten ville have det endnu lettere med at betjene sine kunder, så vil der altid være muligt at forbedre alle softwaresystemer - også Bookie. 
Trods det, så kan ekspedienten nu betjene sine kunder, og Bookie gør det godt.






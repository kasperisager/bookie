\chapter{Konklusion}

Alting bør have en ende. At starte med intet andet end den blå luft, til at sidde med et (næsten) færdigt produkt, er en super følelse. Som om man har hjulpet en person med at løse sit arbejde uden problemer - lækkert. Men når en ekspedient råber om hjælp, fordi han/hun ikke kan betjene sine kunder, så er der et problem. Det er her at Bookie kommer ind i billedet.

I samarbejde med ORM er \textit{Donkey} lavet, og det er blevet til en fornøjelse at arbejde med SQL. Databaser gør dét de er bedst til, og al injicering er umulig, hvilket reducérer de tidlige problemer med Bookie. Der har vores tests gjort deres arbejde godt.
Programmet virker, dog kan der godt tilføjes nogle ændringer, for at få det til at virke optimalt. Nævner nogle af de ændringer, som ville være bedst for Bookie:

\begin{itemize}
  \item Det skal ikke være muligt at indtaste ugyldige telefonnumre, som f.eks. \textit{
  11222222} eller \textit{11400000}.
  %I dette tilfælde ville det være alarmcentralen og politiet, som det ville   gå ud over. Chancen for at nogen ville brokke sig, er rimelig stor. Hvilket ikke havde været så godt.
  \item At se hvor mange ledige pladser der er tilbage, ville være en stor hjælp for ekspedienten.
  \item Mulighed for at vælge sæder mere effektivt.
  % F.eks. at indtaste det antal af sæder man vil reservere, hvorefter det antal af sæder ville blive markeret, og man derefter ville kunne vælge hvilken placering de skulle have.
  
  \item Tastaturunderstøttelse.
  %I tabellerne er det muligt at navigere med piletasterne, da dette kommer automatisk med JavaFX, men f.eks. at vælge sæder med pilene er indtil videre ikke muligt.
  
\end{itemize}
Selv om disse ændringer ville gøre Bookie til et bedre system, og ekspedienten ville have det endnu lettere med at betjene sine kunder, vil der altid være muligt at få et system til at blive bedre, så på et tidspunkt bliver man nødt til at sige stop.

Hvilket viser os, at alting ikke har en ende. Det er altid muligt at forbedre softwaresystemer. Trods alt, så klarer Bookie det godt, og vores ekspedien kan gøre sit arbejde - med tilfredse kunder.
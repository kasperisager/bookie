\chapter{Forord}


Denne rapport er udviklet i desember 2014 på IT-Universitet i København,  under vejledning af Jesper Wendel Madsen. Rapporten er skrevet som 1. semesters hovedprojekt, og er udformet ud fra et softwaresystem.

I projektet er der udviklet et Java-program, med det formål at det skal være muligt at håndtere reservationer af billetter for en biograf. Programmet, kaldt for Bookie, reserverer valgte billetter, redigerer dem, og sletter dem hvis nødvendigt. Parallelt med Bookie er der udviklet en database abstraktion, som kan håndtere data for os.

Bookie er lavet til gavn for en ekspedient, hvis kunder kan henvende sig til. Rapporten derefter, er skrevet til vedkommende, som skal vurdere systemet, og selve brugeren. Også er den skrevet for at få et teknisk perspektiv på hele projektet, hvor der skal være muligt for brugeren at læse den, og derefter forstå hvordan Bookie fungerer.

Rapporten beskriver hvordan projektet er udformet, hvilke metoder er brugt, problemer og eventuelle forbedringer.

Til sidst siger projektgruppen tak til alle der har afprøvet og givet kritik til Bookie.






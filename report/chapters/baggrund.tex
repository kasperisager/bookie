\chapter{Baggrund og problemstilling}
\label{chapter:baggrund-og-problemstilling}

En ekspedient skal med et reservationssystem kunne betjene billetlugen og telefonen i en mindre biograf. Ekspedienten skal kunne reservere billetter til forestillinger samt lave ændringer på eller slette eksisterende reservationer. Al data håndteret af systemet skal gemmes i en relationsdatabase, således at reservationer ikke går tabt.

\section{Brugerhistorier}

Følgende brugerhistorier (\cite{wiki:user-story}) er udarbejdet som krav til reservationssystemet, og tager udgangspunkt i ekspedientens, såvel som kundens rolle i reservationsprocessen.

\subsection{Ekspedientens rolle}

\begin{itemize}
  \item Som en ekspedient vil jeg reservere en eller flere billetter til den forestilling, som en kunde måtte ønske.
  \item Som en ekspedient vil jeg sortere i forestillingerne, så jeg kan finde den forestilling, som en kunde måtte ønske.
  \item Som en ekspedient vil jeg se hvilke sæder er ledige til en given forestilling, så jeg kan finde de sæder, som en kunde måtte ønske.
  \item Som en ekspedient vil jeg knytte en kundes telefonnummer til en reservation, så jeg senere kan finde den igen.
  \item Som en ekspedient vil jeg søge på en kundes telefonnummer, så jeg kan finde deres reservationer igen.
  \item Som en ekspedient vil jeg slette en kundes reservation, hvis de ønsker det.
  \item Som en ekspedient vil jeg ændre en kundes reservation, hvis de ønsker det.
\end{itemize}
  
\subsection{Kundens rolle}

\begin{itemize}
  \item Som en kunde vil jeg se en bestemt eller vilkårlig film på en bestemt eller vilkårlig dag samt et bestemt eller vilkårligt tidspunkt af dagen.
  \item Som en kunde vil jeg have de ledige sæder, som jeg bedst synes om.
  \item Som en kunde vil jeg afreservere mine billetter hvis jeg ikke kan komme til forestillingen.
  \item Som en kunde vil jeg tilføje billetter til fra min reservation hvis nødvendigt.
  \item Som en kunde vil jeg fjerne billetter fra min reservation hvis nødvendigt.
\end{itemize}

Der er en chance for, at der ikke er nok billetter til den valgte forestilling, eller at der ikke er en forestilling på det valgte tidspunkt. Også har vi det eksempel, at hvis en kunde ringer og spørger efter 10 billetter til en bestemt forestilling, hvor der er pladser nok, men pladserne er desværre ikke placeret ved siden af hinanden, hvilke muligheder er der så? Disse problemer kan visuelt afkodes af brugergrænsefladen. Hvis biografen ikke kan udbyde kundens efterspørgsel, så vil der ikke ske andet, end at kunden må finde en anden forestilling eller acceptere, at han/hun ikke kommer i biograf på det valgte tidspunkt.

Brugerhistorierne bliver alle løst på forskellige måder. For det første må der eksistere nogle forestillinger, sale, rækker og sæder før der kan laves nogle reservationer. Forestillingerne i sig selv er en kombination af film, sal, tidspunkt og dag. Disse forskellige elementer skal kunne ses på brugergrænsefladen. Da rækkerne og sæderne skal kunne reserveres til kunder, må der ikke være muligt for ekspedienten at dobbeltreservere.

For det andet skal man bl.a. kunne koble et kunde-id (i form af telefonnummer) sammen med en forestilling, hvorefter disse informationer skal kunne gemmes, hentes igen, evt. redigeres og slettes. Databaser bliver her et nødvendigt redskab.

Der er taget det valg, at bruge et klient-tungt system, som inkluderer at det godt kan bruges uafhængigt af en server. Der er ikke et behov for forbindelse til hverken en server eller et netværk. Al data er placeret lokalt, og bl.a. fordi vi får givet, at biografen kun skal have én billetluge og tillader ikke reservationer over nettet. Dette inkluderer, at det ikke er muligt at lave forskellige opdateringer til databasen på samme tid.

Det færdige projekt skal kunne håndtere alle krav, som er blevet stillede. Det skal kunne tilfredsstille kunden og ekspedienten således, at der hverken er problemer eller mangel på muligheder vedrørende reservationer eller redigering af dem.

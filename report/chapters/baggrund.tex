\chapter{Baggrund og problemstilling}

En ekspedient skal kunne betjene billetlugen og telefonen i en mindre biograf. Opgaven er stillet sådan, at ekspedienten skal kunne betjene begge former for kundehenvendelse uden brug af . Ekspedienten skal kunne reservere pladser/billetter til forskellige forestillinger, lave ændringer på dem, men ikke sælge dem. Et softwaresystem er blevet udviklet til at løse visse krav og brugsscenarier, som er stillet i opgaven - og som gør det lettere for ekspedienten at fuldføre sit arbejde.

De arbejdsopgaver som vores ekspedient skal kunne løse er:
\begin{itemize}
  \item En kunde skal kunne bestille så mange billetter vedkommende har lyst til.
  \item Billetterne skal være knyttet til vilkårlige eller bestemte sæder
  \item Ekspedienten skal kunne fjerne, ændre og tilføje reservationer.
  \item Mulighed for at vælge én bestemt forestilling eller vilkårlige forestillinger.
  \item Mulighed for at vælge ét bestemt tidspunkt eller vilkårlige tidspunkter i løbet af den periode, de forskellige forestillinger bliver vist.
\end{itemize}

Der er en chance for, at der ikke er nok billetter til den valgte forestilling, eller at der ikke er en forestilling på det valgte tidspunkt. Også har vi det eksempel, at hvis en kunde ringer og spørger efter 10 billetter til en bestemt forestilling, hvor der er pladser nok, men pladserne er desværre ikke placeret ved siden af hinanden, hvilke muligheder er der så? Disse problemer kan visuelt afkodes af brugergrænsefladen. Hvis biografen ikke kan udbyde kundens efterspørgsel, så vil der ikke ske andet, end at kunden må finde en anden forestilling eller acceptere, at han/hun ikke kommer i biograf på det valgte tidspunkt.

Brugsscenarierne bliver alle løst på forskellige måder. For det første må der eksistere nogle forestillinger, sale, rækker og sæder før der kan laves nogle reservationer. Forestillingerne i sig selv er en kombination af film, sal, tidspunkt og dag. Disse forskellige elementer skal kunne ses på brugergrænsefladen. Da rækkerne og sæderne skal kunne resereveres til kunder, må der ikke være muligt for ekspedienten at dobbeltreservere.

For det andet skal man bl.a. kunne koble et kunde-id (i form af telefonnummer) sammen med en forestilling, hvorefter disse informationer skal kunne gemmes, hentes igen, evt. redigeres og slettes. Databaser bliver her et nødvendigt redskab.

Da vi har et telefonnummer, som er forbundet til en forestilling, en forestilling der er forbundet til en sal og en film, og en billet der er forbundet til reservation, så har vi kun det der hedder et "one-to-many"-forhold. Ud over det, så fungerer forholdet også kun den ene vej. F.eks. er det kun forestillingen der er forbundet til en sal og en film, hvor filmen og salen ikke er forbundet til forestillingen. Der er altså mulighed for, at både have en film uden at have en forestilling, og en sal uden at have en forestilling. På samme måde fungerer det med telefonnummeret og billetten.

Der er taget det valg, at bruge et klient-tungt system, som inkluderer at det godt kan bruges uafhængigt af en server. Der er ikke et behov for forbindelse til hverken en server eller et netværk. Al data er placeret lokalt, og bl.a. fordi vi får givet, at biografen kun skal have én billetluge og tillader ikke reservationer over nettet. Dette inkluderer, at det ikke er muligt at lave forskellige opdateringer til databasen på samme tid. Her er SQLite en optimal mulighed i forhold til MYSQL og PostgreSQL.

\chapter{Baggrund og problemstilling}



Der er visse brugsscenarier: En kunde skal kunne bestille så mange billetter, vedkommende har lyst til. Disse billetter kan være knyttet til vilkårlige eller bestemte sæder - de skal kunne vælges. Udover dette, skal ekspedienten kunne fjerne, ændre og tilføje reservationer. Det skal være muligt at vælge én bestemt forestilling, vilkårlige forestillinger, ét bestemt tidspunkt og vilkårlige tidspunkter i løbet af den periode de forskellige forestillinger bliver vist.
Der er en chance for, at der ikke er nok billetter til den valgte forestilling, eller at der ikke er en forestilling på det valgte tidspunkt. Disse problemer kan visuelt afkodes af brugergrænsefladen.



Der er taget det valg, at bruge et klient-tungt system, da 



Her gives en omtale af den problemstilling (administrativ, biologisk, økonomisk, . . .) som programmet
behandler, og der opstilles krav til programmet. Der skal være baggrund og forklaring nok til at man kan
forstå hvad det er programmet skal gøre, og hvordan det kan gøre det: formler der skal beregnes, osv. Her
kan være henvisninger til relevant faglig litteratur.
I en større opgave kan dette afsnit opdeles i flere afsnit, f.eks. Baggrund (hvad er problemområdet?) og
Problemformulering (hvad skal programmet kunne?).